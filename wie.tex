\note{
Log. Reduktion einfach. Dann Algorithmus von Daniel Wiebking.

Normalisator des Sockels schwerer.
}

\begin{frame}{Schwach kanonische Form}
\TODO{Erklärung}

\TODO{Satz: kann in nearly-linear time berechnet werden}
\end{frame}

\note{das Beste zum Schluss.
Jeder der mal was mit Permutationsgruppen zu tun hatte, weiß, dass diese
schnell "messy" werden können.
Um WCF zu berechnen benutzen wir perm mors}

\begin{frame}{Permutationsmorphismen}
\TODO{CD aus Diss Seite 44}

\begin{block}{Satz}
Permutationsgruppen zusammen mit Permutationsmorphismen und ``vertikaler''
Verknüpfung bilden eine Kategorie $\mathbf{PermGrp}$.
\end{block}
\end{frame}

\begin{frame}{Permutationsmorphismen: Beispiel}
Beispiel?

oder

Satz 4.3.9 aus Diss?

oder beides?
\end{frame}


\begin{frame}{Permutationsmorphismen: Produkte}
\begin{block}{Satz}
$\mathbf{PermGrp}$ enthält alle endlichen Produkte.
\end{block}

\begin{block}{Satz}
Sei $\Omega$ mit Abbildungen $(p_i : \Omega \to \Omega_i)$ ein Produkt in
$\mathbb{Set}$.
Produkt kompatibel mit $G$.

$P_i$ die ``Fortsetzungen'' von $p_i$ zu perm epis von $G$.

Dann $G$ mit $(P_i)$ Produkt iff. \ldots
\end{block}
\end{frame}

\begin{frame}[standout]
Danke!
\end{frame}
