\begin{frame}{Gruppen (1)}
Gruppen beschreiben Symmetrien

Anwendungen
\begin{itemize}
\item Versuchsdesign
\item Schwarze Löcher
\item Kryptographie (Verschlüsselung)
\item $\ldots$
\end{itemize}
\end{frame}

\begin{frame}{Gruppen (2)}
Gruppen sind interessant
\begin{itemize}
\item Untersuchung mithilfe von Computern
\item Manche Fragen können einfach beantwortet werden $\ldots$
\item $\ldots$ andere hingegen nur mit großem Aufwand
\end{itemize}
\end{frame}

\begin{frame}{Normalisatoren}

Normalisatoren sind Werkzeuge um Gruppen zu verstehen

\begin{itemize}
\item für beliebige Gruppen schwer zu berechnen
\item es ist kein effizienter Algorithmus bekannt
\end{itemize}
\TODO{Bild vom Langen Johann. Captions und footnotes!}
\note{Vergleich mit Hochhaus: wollen jemanden besuchen und kennen nur den
Nachnamen.
Szenario 1: Klingelschilder sind alphabetisch sortiert
Szenario 2: Klingelschilder sind nicht sortiert
Szenario 3: Klingelschilder sind nicht beschriftet

Berechnen von Normalisatoren bewegt sich zwischen Szenario 2 und 3, je nachdem
welche Gruppe man erwischt hat.

Source:
Bild: BR/Carlo Schindhelm
URL: https://www.br.de/radio/bayern2/sendungen/zeit-fuer-bayern/hochhaus-langer-johann-erlangen-100.html
Datum: 08.07.20
}
\end{frame}

\begin{frame}{Primitive Gruppen (1)}
Idee: Normalisatoren von \emph{primitiven Gruppen} sollten leicht zu berechnen
sein.

Primitive Gruppen sind
\begin{itemize}
\item die Bausteine von \emph{Permutationsgruppen}
\item gut verstanden
\end{itemize}
\end{frame}

\begin{frame}{Resultat}
\textcolor{orange}{Normalisatoren}
\pause
von
\textcolor{blue}{primitiven Gruppen mit nicht-regulären Sockeln}
\pause
können in
\textcolor{green}{polynomieller Zeit}
berechnet werden.

\pause
Hoffnung:
bessere Algorithmen für "zusammengesetzte" Gruppen.
\end{frame}
