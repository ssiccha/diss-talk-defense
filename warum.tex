\begin{frame}{Gruppen (1)}
Gruppen beschreiben Symmetrien

Anwendungen
\begin{itemize}
\item Schwarze Löcher
\item Kryptographie (Verschlüsselung)
\item Versuchsdesign
\item $\ldots$
\end{itemize}

\emph{Disclaimer:} hier nur endliche Gruppen und Mengen!
\end{frame}

\begin{frame}
\TODO{erst neut elm, dann abg, dann beide animationen, dann invbar, \ldots}
\end{frame}

\begin{frame}{Gruppen (2)}
\begin{columns}
\begin{column}{0.4\textwidth}
Vier Eigenschaften
\begin{itemize}
\pause
\item Neutrales Element
\pause
\item Abgeschlossen
\pause
\item Invertierbar
\pause
\item Assoziativ
\end{itemize}
\end{column}
\begin{column}{0.5\textwidth}
\begin{center}
  \animategraphics[]{9}{animation/}{1}{9}
\end{center}
\end{column}
\end{columns}
Gruppe die aus Vertauschungen besteht:
\emph{Permutationsgruppe}.
\end{frame}

\begin{frame}[noframenumbering]{Gruppen (2)}
\begin{columns}
\begin{column}{0.4\textwidth}
Vier Eigenschaften
\begin{itemize}
\pause
\item Neutrales Element
\pause
\item Abgeschlossen
\pause
\item Invertierbar
\pause
\item Assoziativ
\end{itemize}
\end{column}
\begin{column}{0.5\textwidth}
\begin{center}
  \animategraphics[]{12}{animation/}{1}{17}
\end{center}
\end{column}
\end{columns}
Gruppe die aus Vertauschungen besteht:
\emph{Permutationsgruppe}.
\end{frame}

\note{
Im Folgenden immer Permutationsgruppen.
}

\begin{frame}{Gruppen (3)}
Gruppen sind interessant
\begin{itemize}
\item Was sind die kleinsten Bausteine\footnote{math.: einfache Gruppen} der Gruppen?
\item Wie kann ich Gruppen zerlegen und zusammensetzen?
\end{itemize}
\end{frame}

\begin{frame}{Klassifikation einfacher Gruppen}
\TODO{Bild Periodensystem einfacher Gruppen. Von Kay Magaard?}
\end{frame}

\begin{frame}{Experimente}
Untersuchung mithilfe von Computern
\begin{itemize}
\item Manche Fragen können einfach beantwortet werden $\ldots$
\item $\ldots$ andere hingegen nur mit großem Aufwand.
\end{itemize}
\end{frame}

\begin{frame}{Normalisatoren}
Normalisatoren:

\begin{itemize}
\item Werkzeuge um Gruppen zu verstehen
\item für beliebige Gruppen schwer zu berechnen
\pause
\item kein effizienter Algorithmus bekannt
\end{itemize}
\end{frame}

\begin{frame}{Wie schwer?}
\TODO{Bild vom Langen Johann. Captions und footnotes!}
\note{Vergleich mit Hochhaus: wollen jemanden besuchen und kennen nur den
Nachnamen.
Szenario 1: Klingelschilder sind alphabetisch sortiert
Szenario 2: Klingelschilder sind nicht sortiert
Szenario 3: Klingelschilder sind nicht beschriftet

Berechnen von Normalisatoren bewegt sich zwischen Szenario 2 und 3, je nachdem
welche Gruppe man erwischt hat.

Source:
Bild: BR/Carlo Schindhelm
URL: https://www.br.de/radio/bayern2/sendungen/zeit-fuer-bayern/hochhaus-langer-johann-erlangen-100.html
Datum: 08.07.20
}
\end{frame}

\note{Überleitung:
Was macht man, wenn man ein großes Problem nicht lösen kann? Man sucht
sich ein einfacheres.
}

\section{Resultat}
\begin{frame}{Primitive Gruppen}
Eine transitive Permutationsgruppe $G \leq \sym \Omega$ heißt
\emph{imprimitiv},
\pause
falls es eine nicht-triviale
$G$-invariante Partition von $\Omega$ gibt,
\pause
sonst \emph{primitiv}.

\pause
\TODO{Bild Eier}
\end{frame}

\note{
Jede Permutationsgruppe lässt sich in primitive zerlegen.
Bausteine der Permutationsgruppen
}

%\begin{frame}{Resultat (1)}
%\textcolor{orange}{Normalisatoren}
%\pause
%von
%\textcolor{blue}{primitiven Gruppen mit nicht-regulären Sockeln}
%\pause
%können in
%\textcolor{green}{polynomieller Zeit}
%berechnet werden.
%\end{frame}

\begin{frame}{Resultat}
    Sei \textcolor{blue}{$G = \gen X \leq \sym \Omega$ eine primitive Gruppe
    mit nicht-regulärem Sockel}.
    \pause
    Dann kann ein
    \textcolor{orange}{Erzeugendensystem von $N_{\sym \Omega}(G)$}
    \pause
    in
    \textcolor{green}{Zeit polynomiell beschränkt in $\abs X$ und $\abs
    \Omega$} bestimmt werden.

\pause
Hoffnung:
bessere Algorithmen für "zusammengesetzte" Gruppen.
\end{frame}

\section{Strategie}

\begin{frame}{Permutationsisomorphie}
\begin{block}{Notation}
\begin{itemize}
\item
$\Omega$, $\Delta$ bezeichnen Mengen
\item
$\sym \Omega$:
Gruppe aller Permutationen von $\Omega$
\item
$\alpha ^ g := g(\alpha)$
\end{itemize}
\end{block}

Zwei Gruppen $G \leq \sym \Omega$ und $H \leq \sym \Delta$ heißen
\emph{permutationsisomorph}, wenn
Isomorphismen $f : \Omega \to \Delta$ und $\phi : G \to H$ existieren,
sodass
$\forall g \in G, \omega \in \Omega$:
\[
    f( \omega ^ g ) = f(\omega) ^ {\phi(g)}.
\]
\end{frame}

\begin{frame}{Struktursatz primitiver Gruppen}
Der \emph{Sockel} einer Gruppe $G$, geschrieben
$\soc G$, ist das Erzeugnis aller minimalen Normalteiler.

Der Sockel einer primitiven Gruppe ist charakteristisch einfach.

\begin{block}{Satz von O'Nan-Scott}
Sei $G \leq \sym \Omega$ eine primitive Permutationsgruppe.
Bis auf Permutationsisomorphie kennen wir
\begin{itemize}
\item $\soc G$,
\item $N_{\sym \Omega}( \soc G )$.
\end{itemize}
\end{block}
\end{frame}

\note{Sockel charakteristisch einfach $\Rightarrow$ Rekursion zu einfachen
Gruppen}

\begin{frame}{Strategie}
Sei $G \leq \sym \Omega$ eine primitive Permutationsgruppe.

Sei $M := N_{\sym \Omega}(\soc G)$.
Es gilt $N_{\sym \Omega}(G) = N_M(G)$.

Bestimme
\begin{itemize}
\item $M := N_{\sym \Omega}(\soc G)$.
\item Gruppenhom.
$
    \rho : M \to S_k
$
mit $k \in O(\log \abs \Omega)$
und
\[
    N_{\sym \Omega}(G) = N_{M}(G) = \rho ^ {-1}( N_{\rho(M)}(\rho(G)) ).
\]
\end{itemize}
\end{frame}

\note{
\begin{itemize}
\item
Perm-iso in kanonische Gestalt rekonstruieren.
Dann Klassifikation endlicher einfacher Gruppen anwenden.
\item
log. Reduktion: aus einem produktoperation Kranzprodukt mache ein imprimitives
Kranzprodukt.
\end{itemize}
}
