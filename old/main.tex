 \documentclass{beamer}
%\documentclass[handout]{beamer}
\usetheme{metropolis}

%\setbeameroption{show notes}
%\setbeameroption{show notes on second screen}

% Packages
\usepackage{animate}

\usepackage{appendixnumberbeamer}
\usepackage{xcolor}
\usepackage{multirow}
% Maths
\usepackage{amsthm}
\usepackage{tikz}
\usetikzlibrary{cd}

% Package Settings
% AMSTHM
\theoremstyle{plain}
\newtheorem{thm}{Theorem}[section]
\newtheorem{mylemma}[thm]{Lemma}
\newtheorem{cor}[thm]{Corollary}
\newtheorem{rem}[thm]{Remark}
\newtheorem{alg}[thm]{Algorithm}

\theoremstyle{definition}
\newtheorem{defn}[thm]{Definition}

% BEAMER
% Hack for only and tabular
\newcommand{\tabularonly}{AS}
% Hack for tilde and hat
\newcommand{\mytilde}[1]    {\ensuremath{\tilde{#1}}}
\newcommand{\myhat}[1]    {\ensuremath{\widehat{#1}}}
% Hack to display \Gamma with a little bit of space to the left
% Otherwise \ind _ \Gamma ^ H looks really weird
\let\oldGamma\Gamma
\renewcommand{\Gamma}{{\hspace{0.5pt} \oldGamma}}
% Hack to reference the notation-definition in "lift a.s. normaliser"
\newcommand{\refdef}    {{(6.1)}}

% Symbols
% Write function definitions like
% f \from A \to B
\newcommand{\from}	{\ensuremath{\colon}}
\renewcommand{\to}	{\ensuremath{\rightarrow}}
\newcommand{\into}	{\ensuremath{\hookrightarrow}}
\newcommand{\onto}	{\ensuremath{\twoheadrightarrow}}
\newcommand{\iso}	{\ensuremath{\xrightarrow{\,\raisebox{-1pt}{\ensuremath{\scriptstyle{\sim}}}\,}}}
\newcommand{\N}	{\ensuremath{\mathbb N}}
\newcommand{\Z}	{\ensuremath{\mathbb Z}}
\newcommand{\Q}	{\ensuremath{\mathbb Q}}
\newcommand{\Sn}{{\ensuremath{\mathrm{S}_n}}}
\newcommand{\uln} {{\ensuremath{\underline n}}}

%%% Categories
\newcommand{\permgrp}   {\ensuremath{\mathbf{PermGrp}}}

%%% Unary Operators
% Functions
\DeclareMathOperator{\dom}{dom}
\DeclareMathOperator{\id}{id}
\DeclareMathOperator{\im}{Im}
% Groups
\DeclareMathOperator{\sym}{Sym}
\DeclareMathOperator{\alternating}{Alt}
\DeclareMathOperator{\soc}{soc}
\DeclareMathOperator{\aut}{Aut}
\DeclareMathOperator{\out}{Out}
\DeclareMathOperator{\ind}{Ind}
\DeclareMathOperator{\res}{Res}
\DeclareMathOperator{\stab}{Stab}
\newcommand{\hatOperator}   {\ensuremath{\, \myhat \cdot\,}}
\def\Norm#1#2{\mathrm{N}_{#1}(#2)}

%%% Binary Operators %%%
\newcommand{\union}{\mathbin{\cup}}
\newcommand{\bigunion}{\mathbin{\bigcup}}
\newcommand{\disjointunion}{\mathbin{\uplus}}
\newcommand{\intersection}{\mathbin{\cap}}
\newcommand{\bigintersection}{\mathbin{\bigcap}}
\newcommand{\norm}{\mathbin{\lhd}}
\newcommand{\characteristic}{\mathbin{\text{char}}}


% New commands
\newcommand{\abs}[1]	{%
	\ensuremath{
		\left| #1 \right|
	}%
}
\newcommand{\gen}[1]	{
	\ensuremath{
		\left\langle \, #1 \, \right\rangle
	}
}
\newcommand{\set}[1]	{
	\ensuremath{
		\left\{ #1 \right\}
	}
}
\newcommand{\sset}[2] {
	\ensuremath{
		\left\{ \left.\,
			#1
		~\right|~
			#2
		\,\right\}
	}
}


\title{Normalisers in Quasipolynomial Time and \\
the Category of Permutation Groups}
\date{May 9, 2019}
\author{Sergio Siccha}
\institute{Lehr- und Forschungsgebiet Algebra, RWTH Aachen}

\begin{document}

\maketitle
\note{
VERSION WITH NOTES!
}

\section{Introduction}
\begin{frame}{Goal}
    \begin{thm}
        Let $G = \gen{X} \leq \sym \Omega$ be a primitive group
        of {\color{blue} PA type}.
        The normaliser
        $N_{\sym \Omega}(G)$
        can be computed in {\color{blue} quasipolynomial} time
        $O(n ^ 3 \cdot 2 ^ {2 \log n \log \log n} \cdot \abs X)$.
    \end{thm}

    \vspace{1em}
    \pause
    Joint work with Prof. Colva Roney-Dougal.
\end{frame}

\begin{frame}{Recursion for Normalisers}
    \begin{center}
        \hspace{-5em}
        \begin{tabular}{r c}
            & Intransitive
            \\
            Mun See Chang & $\updownarrow$
            \\
            & Transitive
            \\
            & $\updownarrow$
            \\
            & Primitive
            \\
            Me & $\updownarrow$
            \\
            & Simple
        \end{tabular}
    \end{center}
\end{frame}

\begin{frame}{Conventions}
\begin{itemize}
\setlength\itemsep{1em}
\item
$\log := \log_2$.
\pause
\item
All groups and sets are finite!
\pause
\item
$\Omega, \Delta$ denote sets,
$G, H, T$ denote groups.
\pause
\begin{itemize}
    \item
    $T$ \emph{always} denotes a finite non-abelian simple group.
    \pause
\end{itemize}
\item
Functions act from the left $f(x)$
but groups from the right: $\alpha ^ g = g(\alpha)$.
\end{itemize}
\end{frame}



\begin{center}
  \animategraphics[autoplay,loop]{5}{animation/rot-}{1}{10}
\end{center}

\end{document}
