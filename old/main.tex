 \documentclass{beamer}
%\documentclass[handout]{beamer}
\usetheme{metropolis}

%\setbeameroption{show notes}
%\setbeameroption{show notes on second screen}

% Packages
\usepackage{animate}

\usepackage{appendixnumberbeamer}
\usepackage{xcolor}
\usepackage{multirow}
% Maths
\usepackage{amsthm}
\usepackage{tikz}
\usetikzlibrary{cd}

% Package Settings
% AMSTHM
\theoremstyle{plain}
\newtheorem{thm}{Theorem}[section]
\newtheorem{mylemma}[thm]{Lemma}
\newtheorem{cor}[thm]{Corollary}
\newtheorem{rem}[thm]{Remark}
\newtheorem{alg}[thm]{Algorithm}

\theoremstyle{definition}
\newtheorem{defn}[thm]{Definition}

\input{./abbreviations.tex}

\title{Normalisers in Quasipolynomial Time and \\
the Category of Permutation Groups}
\date{May 9, 2019}
\author{Sergio Siccha}
\institute{Lehr- und Forschungsgebiet Algebra, RWTH Aachen}

\begin{document}

\maketitle
\note{
VERSION WITH NOTES!
}

\section{Introduction}
\begin{frame}{Goal}
    \begin{thm}
        Let $G = \gen{X} \leq \sym \Omega$ be a primitive group
        of {\color{blue} PA type}.
        The normaliser
        $N_{\sym \Omega}(G)$
        can be computed in {\color{blue} quasipolynomial} time
        $O(n ^ 3 \cdot 2 ^ {2 \log n \log \log n} \cdot \abs X)$.
    \end{thm}

    \vspace{1em}
    \pause
    Joint work with Prof. Colva Roney-Dougal.
\end{frame}

\begin{frame}{Recursion for Normalisers}
    \begin{center}
        \hspace{-5em}
        \begin{tabular}{r c}
            & Intransitive
            \\
            Mun See Chang & $\updownarrow$
            \\
            & Transitive
            \\
            & $\updownarrow$
            \\
            & Primitive
            \\
            Me & $\updownarrow$
            \\
            & Simple
        \end{tabular}
    \end{center}
\end{frame}

\begin{frame}{Conventions}
\begin{itemize}
\setlength\itemsep{1em}
\item
$\log := \log_2$.
\pause
\item
All groups and sets are finite!
\pause
\item
$\Omega, \Delta$ denote sets,
$G, H, T$ denote groups.
\pause
\begin{itemize}
    \item
    $T$ \emph{always} denotes a finite non-abelian simple group.
    \pause
\end{itemize}
\item
Functions act from the left $f(x)$
but groups from the right: $\alpha ^ g = g(\alpha)$.
\end{itemize}
\end{frame}



\begin{center}
  \animategraphics[autoplay,loop]{5}{animation/rot-}{1}{10}
\end{center}

\end{document}
